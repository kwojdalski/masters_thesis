\documentclass{article}\usepackage[]{graphicx}\usepackage[]{color}
%% maxwidth is the original width if it is less than linewidth
%% otherwise use linewidth (to make sure the graphics do not exceed the margin)
\makeatletter
\def\maxwidth{ %
  \ifdim\Gin@nat@width>\linewidth
    \linewidth
  \else
    \Gin@nat@width
  \fi
}
\makeatother

\definecolor{fgcolor}{rgb}{0.514, 0.58, 0.588}
\newcommand{\hlnum}[1]{\textcolor[rgb]{0.863,0.196,0.184}{#1}}%
\newcommand{\hlstr}[1]{\textcolor[rgb]{0.863,0.196,0.184}{#1}}%
\newcommand{\hlcom}[1]{\textcolor[rgb]{0.345,0.431,0.459}{#1}}%
\newcommand{\hlopt}[1]{\textcolor[rgb]{0.576,0.631,0.631}{#1}}%
\newcommand{\hlstd}[1]{\textcolor[rgb]{0.514,0.58,0.588}{#1}}%
\newcommand{\hlkwa}[1]{\textcolor[rgb]{0.796,0.294,0.086}{#1}}%
\newcommand{\hlkwb}[1]{\textcolor[rgb]{0.522,0.6,0}{#1}}%
\newcommand{\hlkwc}[1]{\textcolor[rgb]{0.796,0.294,0.086}{#1}}%
\newcommand{\hlkwd}[1]{\textcolor[rgb]{0.576,0.631,0.631}{#1}}%

\usepackage{framed}
\makeatletter
\newenvironment{kframe}{%
 \def\at@end@of@kframe{}%
 \ifinner\ifhmode%
  \def\at@end@of@kframe{\end{minipage}}%
  \begin{minipage}{\columnwidth}%
 \fi\fi%
 \def\FrameCommand##1{\hskip\@totalleftmargin \hskip-\fboxsep
 \colorbox{shadecolor}{##1}\hskip-\fboxsep
     % There is no \\@totalrightmargin, so:
     \hskip-\linewidth \hskip-\@totalleftmargin \hskip\columnwidth}%
 \MakeFramed {\advance\hsize-\width
   \@totalleftmargin\z@ \linewidth\hsize
   \@setminipage}}%
 {\par\unskip\endMakeFramed%
 \at@end@of@kframe}
\makeatother

\definecolor{shadecolor}{rgb}{.97, .97, .97}
\definecolor{messagecolor}{rgb}{0, 0, 0}
\definecolor{warningcolor}{rgb}{1, 0, 1}
\definecolor{errorcolor}{rgb}{1, 0, 0}
\newenvironment{knitrout}{}{} % an empty environment to be redefined in TeX

\usepackage{alltt}
\usepackage{geometry}
\geometry{verbose, tmargin=2.5cm, bmargin=2.5cm, lmargin=2.5cm, rmargin=2.5cm}
\IfFileExists{upquote.sty}{\usepackage{upquote}}{}
\begin{document}

\title{}
\author{}
\maketitle





\subsubsection{genNavbar}

\texttt{genNavbar} generates a navigation bar for a web page.
The html file created is generally written to the project's \texttt{docs/Rmd/include} directory.
However, if this function is used to create a navbar for a Github user web page, the html file should be stored in a sensible location inside the user pages repository, e.g., \texttt{leonawicz.github.io/assets}.

The common navigation bar html is included at the beginning of the body of the html for each web page in the project's website.
\texttt{menu} is a vector of names for each dropdown menu.
\texttt{submenus} is a list of vectors of menu options corresponding to each menu.
\texttt{files} is a similar list of vectors.
Each element is either an html file for a web page to be associated with the submenu link,
"header" to indicate the corresponding name in \texttt{submenus} is only a group label and not a link to a web page,
or "divider" to indicate placement of a bar for separating groups in a dropdown menu.

\texttt{theme} can be either \texttt{united} (default) or \texttt{cyborg}.
Both are from Bootswatch.
The function must apply some internal differentiation in the construction of the html navigation bar between themes.
This is currently the only \texttt{rpm} function which attempts to handle multiple Bootswatch themes with different CSS tags.

\begin{knitrout}
\definecolor{shadecolor}{rgb}{0, 0.169, 0.212}\color{fgcolor}\begin{kframe}
\begin{alltt}
\hlcom{# Functions for Github project websites}
\hlstd{genNavbar} \hlkwb{<-} \hlkwa{function}\hlstd{(}\hlkwc{htmlfile} \hlstd{=} \hlstr{"navbar.html"}\hlstd{,} \hlkwc{title}\hlstd{,} \hlkwc{menu}\hlstd{,} \hlkwc{submenus}\hlstd{,} \hlkwc{files}\hlstd{,}
    \hlkwc{title.url} \hlstd{=} \hlstr{"index.html"}\hlstd{,} \hlkwc{home.url} \hlstd{=} \hlstr{"index.html"}\hlstd{,} \hlkwc{site.url} \hlstd{=} \hlstr{""}\hlstd{,} \hlkwc{site.name} \hlstd{=} \hlstr{"Github"}\hlstd{,}
    \hlkwc{theme} \hlstd{=} \hlstr{"united"}\hlstd{,} \hlkwc{include.home} \hlstd{=} \hlnum{FALSE}\hlstd{) \{}
    \hlkwa{if} \hlstd{(}\hlopt{!}\hlstd{(theme} \hlopt \hlkwd{c}\hlstd{(}\hlstr{"united"}\hlstd{,} \hlstr{"cyborg"}\hlstd{)))}
        \hlkwd{stop}\hlstd{(}\hlstr{"Only the following themes supported: united, cyborg."}\hlstd{)}

    \hlstd{navClassStrings} \hlkwb{<-} \hlkwa{function}\hlstd{(}\hlkwc{x}\hlstd{) \{}
        \hlkwd{switch}\hlstd{(x,} \hlkwc{united} \hlstd{=} \hlkwd{c}\hlstd{(}\hlstr{"brand"}\hlstd{,} \hlstr{"nav-collapse collapse"}\hlstd{,} \hlstr{"nav"}\hlstd{,} \hlstr{"nav pull-right"}\hlstd{,}
            \hlstr{"navbar-inner"}\hlstd{,} \hlstr{"container"}\hlstd{,} \hlstr{""}\hlstd{,} \hlstr{"btn btn-navbar"}\hlstd{,} \hlstr{".nav-collapse"}\hlstd{,}
            \hlstr{"</div>\textbackslash{}n"}\hlstd{),} \hlkwc{cyborg} \hlstd{=} \hlkwd{c}\hlstd{(}\hlstr{"navbar-brand"}\hlstd{,} \hlstr{"navbar-collapse collapse navbar-responsive-collapse"}\hlstd{,}
            \hlstr{"nav navbar-nav"}\hlstd{,} \hlstr{"nav navbar-nav navbar-right"}\hlstd{,} \hlstr{"container"}\hlstd{,} \hlstr{"navbar-header"}\hlstd{,}
            \hlstr{"      </div>\textbackslash{}n"}\hlstd{,} \hlstr{"navbar-toggle"}\hlstd{,} \hlstr{".nav-collapse"}\hlstd{,} \hlstr{""}\hlstd{))}
    \hlstd{\}}

    \hlstd{ncs} \hlkwb{<-} \hlkwd{navClassStrings}\hlstd{(theme)}

    \hlstd{fillSubmenu} \hlkwb{<-} \hlkwa{function}\hlstd{(}\hlkwc{x}\hlstd{,} \hlkwc{name}\hlstd{,} \hlkwc{file}\hlstd{,} \hlkwc{theme}\hlstd{) \{}
        \hlkwa{if} \hlstd{(theme} \hlopt{==} \hlstr{"united"}\hlstd{)}
            \hlstd{dd.menu.header} \hlkwb{<-} \hlstr{"nav-header"} \hlkwa{else if} \hlstd{(theme} \hlopt{==} \hlstr{"cyborg"}\hlstd{)}
            \hlstd{dd.menu.header} \hlkwb{<-} \hlstr{"dropdown-header"}
        \hlkwa{if} \hlstd{(file[x]} \hlopt{==} \hlstr{"divider"}\hlstd{)}
            \hlkwd{return}\hlstd{(}\hlstr{"              <li class=\textbackslash{}"divider\textbackslash{}"></li>\textbackslash{}n"}\hlstd{)}
        \hlkwa{if} \hlstd{(file[x]} \hlopt{==} \hlstr{"header"}\hlstd{)}
            \hlkwd{return}\hlstd{(}\hlkwd{paste0}\hlstd{(}\hlstr{"              <li class=\textbackslash{}""}\hlstd{, dd.menu.header,} \hlstr{"\textbackslash{}">"}\hlstd{,}
                \hlstd{name[x],} \hlstr{"</li>\textbackslash{}n"}\hlstd{))}
        \hlkwd{paste0}\hlstd{(}\hlstr{"              <li><a href=\textbackslash{}""}\hlstd{, file[x],} \hlstr{"\textbackslash{}">"}\hlstd{, name[x],} \hlstr{"</a></li>\textbackslash{}n"}\hlstd{)}
    \hlstd{\}}

    \hlstd{fillMenu} \hlkwb{<-} \hlkwa{function}\hlstd{(}\hlkwc{x}\hlstd{,} \hlkwc{menu}\hlstd{,} \hlkwc{submenus}\hlstd{,} \hlkwc{files}\hlstd{,} \hlkwc{theme}\hlstd{) \{}
        \hlstd{m} \hlkwb{<-} \hlstd{menu[x]}
        \hlstd{gs.menu} \hlkwb{<-} \hlkwd{gsub}\hlstd{(}\hlstr{" "}\hlstd{,} \hlstr{"-"}\hlstd{,} \hlkwd{tolower}\hlstd{(m))}
        \hlstd{s} \hlkwb{<-} \hlstd{submenus[[x]]}
        \hlstd{f} \hlkwb{<-} \hlstd{files[[x]]}
        \hlkwa{if} \hlstd{(s[}\hlnum{1}\hlstd{]} \hlopt{==} \hlstr{"empty"}\hlstd{) \{}
            \hlstd{y} \hlkwb{<-} \hlkwd{paste0}\hlstd{(}\hlstr{"<li><a href=\textbackslash{}""}\hlstd{, f,} \hlstr{"\textbackslash{}">"}\hlstd{, m,} \hlstr{"</a></li>\textbackslash{}n"}\hlstd{)}
        \hlstd{\}} \hlkwa{else} \hlstd{\{}
            \hlstd{y} \hlkwb{<-} \hlkwd{paste0}\hlstd{(}\hlstr{"<li class=\textbackslash{}"dropdown\textbackslash{}">\textbackslash{}n            <a href=\textbackslash{}""}\hlstd{, gs.menu,}
                \hlstr{"\textbackslash{}" class=\textbackslash{}"dropdown-toggle\textbackslash{}" data-toggle=\textbackslash{}"dropdown\textbackslash{}">"}\hlstd{, m,}
                \hlstr{" <b class=\textbackslash{}"caret\textbackslash{}"></b></a>\textbackslash{}n            <ul class=\textbackslash{}"dropdown-menu\textbackslash{}">\textbackslash{}n"}\hlstd{,}
                \hlkwd{paste}\hlstd{(}\hlkwd{sapply}\hlstd{(}\hlnum{1}\hlopt{:}\hlkwd{length}\hlstd{(s), fillSubmenu,} \hlkwc{name} \hlstd{= s,} \hlkwc{file} \hlstd{= f,} \hlkwc{theme} \hlstd{= theme),}
                  \hlkwc{sep} \hlstd{=} \hlstr{""}\hlstd{,} \hlkwc{collapse} \hlstd{=} \hlstr{""}\hlstd{),} \hlstr{"            </ul>\textbackslash{}n"}\hlstd{,} \hlkwc{collapse} \hlstd{=} \hlstr{""}\hlstd{)}
        \hlstd{\}}
    \hlstd{\}}

    \hlkwa{if} \hlstd{(include.home)}
        \hlstd{home} \hlkwb{<-} \hlkwd{paste0}\hlstd{(}\hlstr{"<li><a href=\textbackslash{}""}\hlstd{, home.url,} \hlstr{"\textbackslash{}">Home</a></li>\textbackslash{}n          "}\hlstd{)} \hlkwa{else} \hlstd{home} \hlkwb{<-} \hlstr{""}
    \hlstd{x} \hlkwb{<-} \hlkwd{paste0}\hlstd{(}\hlstr{"<div class=\textbackslash{}"navbar navbar-default navbar-fixed-top\textbackslash{}">\textbackslash{}n  <div class=\textbackslash{}""}\hlstd{,}
        \hlstd{ncs[}\hlnum{5}\hlstd{],} \hlstr{"\textbackslash{}">\textbackslash{}n    <div class=\textbackslash{}""}\hlstd{, ncs[}\hlnum{6}\hlstd{],} \hlstr{"\textbackslash{}">\textbackslash{}n      <button type=\textbackslash{}"button\textbackslash{}" class=\textbackslash{}""}\hlstd{,}
        \hlstd{ncs[}\hlnum{8}\hlstd{],} \hlstr{"\textbackslash{}" data-toggle=\textbackslash{}"collapse\textbackslash{}" data-target=\textbackslash{}""}\hlstd{, ncs[}\hlnum{9}\hlstd{],} \hlstr{"\textbackslash{}">\textbackslash{}n        <span class=\textbackslash{}"icon-bar\textbackslash{}"></span>\textbackslash{}n        <span class=\textbackslash{}"icon-bar\textbackslash{}"></span>\textbackslash{}n        <span class=\textbackslash{}"icon-bar\textbackslash{}"></span>\textbackslash{}n      </button>\textbackslash{}n      <a class=\textbackslash{}""}\hlstd{,}
        \hlstd{ncs[}\hlnum{1}\hlstd{],} \hlstr{"\textbackslash{}" href=\textbackslash{}""}\hlstd{, title.url,} \hlstr{"\textbackslash{}">"}\hlstd{, title,} \hlstr{"</a>\textbackslash{}n"}\hlstd{, ncs[}\hlnum{7}\hlstd{],} \hlstr{"      <div class=\textbackslash{}""}\hlstd{,}
        \hlstd{ncs[}\hlnum{2}\hlstd{],} \hlstr{"\textbackslash{}">\textbackslash{}n        <ul class=\textbackslash{}""}\hlstd{, ncs[}\hlnum{3}\hlstd{],} \hlstr{"\textbackslash{}">\textbackslash{}n          "}\hlstd{, home,}
        \hlkwd{paste}\hlstd{(}\hlkwd{sapply}\hlstd{(}\hlnum{1}\hlopt{:}\hlkwd{length}\hlstd{(menu), fillMenu,} \hlkwc{menu} \hlstd{= menu,} \hlkwc{submenus} \hlstd{= submenus,}
            \hlkwc{files} \hlstd{= files,} \hlkwc{theme} \hlstd{= theme),} \hlkwc{sep} \hlstd{=} \hlstr{""}\hlstd{,} \hlkwc{collapse} \hlstd{=} \hlstr{"\textbackslash{}n          "}\hlstd{),}
        \hlstr{"        </ul>\textbackslash{}n        <ul class=\textbackslash{}""}\hlstd{, ncs[}\hlnum{4}\hlstd{],} \hlstr{"\textbackslash{}">\textbackslash{}n          <a class=\textbackslash{}"btn btn-primary\textbackslash{}" href=\textbackslash{}""}\hlstd{,}
        \hlstd{site.url,} \hlstr{"\textbackslash{}">\textbackslash{}n            <i class=\textbackslash{}"fa fa-github fa-lg\textbackslash{}"></i>\textbackslash{}n            "}\hlstd{,}
        \hlstd{site.name,} \hlstr{"\textbackslash{}n          </a>\textbackslash{}n        </ul>\textbackslash{}n      </div><!--/.nav-collapse -->\textbackslash{}n    </div>\textbackslash{}n  "}\hlstd{,}
        \hlstd{ncs[}\hlnum{10}\hlstd{],} \hlstr{"</div>\textbackslash{}n"}\hlstd{,} \hlkwc{collpase} \hlstd{=} \hlstr{""}\hlstd{)}
    \hlkwd{sink}\hlstd{(htmlfile)}
    \hlkwd{cat}\hlstd{(x)}
    \hlkwd{sink}\hlstd{()}
    \hlstd{x}
\hlstd{\}}
\end{alltt}
\end{kframe}
\end{knitrout}

\subsubsection{genOutyaml}

\texttt{genOutyaml} generates the \texttt{\_out.yaml} file for yaml front-matter common to all html files in the project website.
The file should be written to the project's \texttt{docs/Rmd} directory.
\texttt{lib} specifies the library directory for any associated files.
yaml \texttt{includes} for external html common to all project web pages in the site can also be specified with \texttt{header}, \texttt{before\_body}, and \texttt{after\_body}.
These can be specified by file basename only (no path) and the function assumes these files are in the \texttt{docs/Rmd/include} directory.
At this time all external libraries must be provided by the user, for example in \texttt{docs/Rmd/libs}.
It is recommended. See the project repo [gh-pages](https://github.com/leonawicz/ProjectManagement/tree/gh-pages "gh-pages") branch for an example.

\begin{knitrout}
\definecolor{shadecolor}{rgb}{0, 0.169, 0.212}\color{fgcolor}\begin{kframe}
\begin{alltt}
\hlstd{genOutyaml} \hlkwb{<-} \hlkwa{function}\hlstd{(}\hlkwc{file}\hlstd{,} \hlkwc{theme} \hlstd{=} \hlstr{"cosmo"}\hlstd{,} \hlkwc{highlight} \hlstd{=} \hlstr{"zenburn"}\hlstd{,} \hlkwc{lib} \hlstd{=} \hlkwa{NULL}\hlstd{,}
    \hlkwc{header} \hlstd{=} \hlkwa{NULL}\hlstd{,} \hlkwc{before_body} \hlstd{=} \hlkwa{NULL}\hlstd{,} \hlkwc{after_body} \hlstd{=} \hlkwa{NULL}\hlstd{) \{}
    \hlstd{output.yaml} \hlkwb{<-} \hlkwd{paste0}\hlstd{(}\hlstr{"html_document:\textbackslash{}n  self_contained: false\textbackslash{}n  theme: "}\hlstd{,}
        \hlstd{theme,} \hlstr{"\textbackslash{}n  highlight: "}\hlstd{, highlight,} \hlstr{"\textbackslash{}n  mathjax: null\textbackslash{}n  toc_depth: 2\textbackslash{}n"}\hlstd{)}
    \hlkwa{if} \hlstd{(}\hlopt{!}\hlkwd{is.null}\hlstd{(lib))}
        \hlstd{output.yaml} \hlkwb{<-} \hlkwd{paste0}\hlstd{(output.yaml,} \hlstr{"  lib_dir: "}\hlstd{, lib,} \hlstr{"\textbackslash{}n"}\hlstd{)}
    \hlstd{output.yaml} \hlkwb{<-} \hlkwd{paste0}\hlstd{(output.yaml,} \hlstr{"  includes:\textbackslash{}n"}\hlstd{)}
    \hlkwa{if} \hlstd{(}\hlopt{!}\hlkwd{is.null}\hlstd{(header))}
        \hlstd{output.yaml} \hlkwb{<-} \hlkwd{paste0}\hlstd{(output.yaml,} \hlstr{"    in_header: "}\hlstd{, header,} \hlstr{"\textbackslash{}n"}\hlstd{)}
    \hlkwa{if} \hlstd{(}\hlopt{!}\hlkwd{is.null}\hlstd{(before_body))}
        \hlstd{output.yaml} \hlkwb{<-} \hlkwd{paste0}\hlstd{(output.yaml,} \hlstr{"    before_body: "}\hlstd{, before_body,}
            \hlstr{"\textbackslash{}n"}\hlstd{)}
    \hlkwa{if} \hlstd{(}\hlopt{!}\hlkwd{is.null}\hlstd{(after_body))}
        \hlstd{output.yaml} \hlkwb{<-} \hlkwd{paste0}\hlstd{(output.yaml,} \hlstr{"    after_body: "}\hlstd{, after_body,} \hlstr{"\textbackslash{}n"}\hlstd{)}
    \hlkwd{sink}\hlstd{(file)}
    \hlkwd{cat}\hlstd{(output.yaml)}
    \hlkwd{sink}\hlstd{()}
    \hlstd{output.yaml}
\hlstd{\}}
\end{alltt}
\end{kframe}
\end{knitrout}

\end{document}
