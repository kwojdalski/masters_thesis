
  
\documentclass{pracamgr_wne}\usepackage[]{graphicx}\usepackage[]{color}
%% maxwidth is the original width if it is less than linewidth
%% otherwise use linewidth (to make sure the graphics do not exceed the margin)
\makeatletter
\def\maxwidth{ %
  \ifdim\Gin@nat@width>\linewidth
    \linewidth
  \else
    \Gin@nat@width
  \fi
}
\makeatother

\definecolor{fgcolor}{rgb}{0.345, 0.345, 0.345}
\newcommand{\hlnum}[1]{\textcolor[rgb]{0.686,0.059,0.569}{#1}}%
\newcommand{\hlstr}[1]{\textcolor[rgb]{0.192,0.494,0.8}{#1}}%
\newcommand{\hlcom}[1]{\textcolor[rgb]{0.678,0.584,0.686}{\textit{#1}}}%
\newcommand{\hlopt}[1]{\textcolor[rgb]{0,0,0}{#1}}%
\newcommand{\hlstd}[1]{\textcolor[rgb]{0.345,0.345,0.345}{#1}}%
\newcommand{\hlkwa}[1]{\textcolor[rgb]{0.161,0.373,0.58}{\textbf{#1}}}%
\newcommand{\hlkwb}[1]{\textcolor[rgb]{0.69,0.353,0.396}{#1}}%
\newcommand{\hlkwc}[1]{\textcolor[rgb]{0.333,0.667,0.333}{#1}}%
\newcommand{\hlkwd}[1]{\textcolor[rgb]{0.737,0.353,0.396}{\textbf{#1}}}%
\let\hlipl\hlkwb

\usepackage{framed}
\makeatletter
\newenvironment{kframe}{%
 \def\at@end@of@kframe{}%
 \ifinner\ifhmode%
  \def\at@end@of@kframe{\end{minipage}}%
  \begin{minipage}{\columnwidth}%
 \fi\fi%
 \def\FrameCommand##1{\hskip\@totalleftmargin \hskip-\fboxsep
 \colorbox{shadecolor}{##1}\hskip-\fboxsep
     % There is no \\@totalrightmargin, so:
     \hskip-\linewidth \hskip-\@totalleftmargin \hskip\columnwidth}%
 \MakeFramed {\advance\hsize-\width
   \@totalleftmargin\z@ \linewidth\hsize
   \@setminipage}}%
 {\par\unskip\endMakeFramed%
 \at@end@of@kframe}
\makeatother

\definecolor{shadecolor}{rgb}{.97, .97, .97}
\definecolor{messagecolor}{rgb}{0, 0, 0}
\definecolor{warningcolor}{rgb}{1, 0, 1}
\definecolor{errorcolor}{rgb}{1, 0, 0}
\newenvironment{knitrout}{}{} % an empty environment to be redefined in TeX

\usepackage{alltt}
\usepackage{afterpage}

%\usepackage[utf8x]{inputenc}
\usepackage[cp1250]{inputenc}
\usepackage[T1]{fontenc}
\usepackage{helvet}
% \usepackage{fontspec}
% \setmainfont{Arial}
\renewcommand{\familydefault}{\sfdefault}

% \usepackage[
% backend=biber,
% style=authoryear,
% parencitestyle=authoryear
% ]{biblatex}
% \usepackage[style=numeric-comp]{biblatex}
% \usepackage[style=authoryear]{biblatex}
\usepackage[backend=biber,style=authoryear]{biblatex}
\addbibresource{working_papers/library.bib}

\author{Krzysztof Wojdalski}

\nralbumu{310284}

\title{Master's thesis}
  
\tytulang{Master's thesis}

\speciality{Quantitative Finance}


\opiekun{dr Juliusz Jablecki
  Department of Quantitative Finance
}

\date{2016-01-01}


\dziedzina{11.2 Quantitative Finance}

\klasyfikacja{D. Software}

% Slowa kluczowe:
\keywords{FX, forex market, trading, reinforcement learning, Algo trading, artificial intelligence,
    quantitative finance}

\linespread{1.3} % interlinia 1,5 wiersza
%\linespread{1.6} % interlinia 2 wiersze

\newtheorem{defi}{Definicja}[section]
\IfFileExists{upquote.sty}{\usepackage{upquote}}{}
\begin{document}

\maketitle


\begin{summary}

The master's thesis is about the Reinforcement Learning application in the foreign exchange market. 
The author starts with describing the FX market, analyzing market organization, participants, and changes in the last
years. He tries to explain current trends and the possible directions. 
The next part consists of theoretical pattern for the research - description of financial models, 
and the AI algorithms. 
Implementation of the RL-based approach in the third chapter, based on Q-learning, gives spurious results. 


\end{summary}

\tableofcontents



\chapter{Introduction}



% !Rnw root = ../../masters_thesis.Rnw

Financial markets has been interested in computer science methods since the early 1980s. Though there are ways
to gain abnormal, positive returns by following traditional ways of investing, such as buy and hold, more modern methods gain on popularity. 
One of the most popular and emerging category among innovative approaches is the artificial intelligence-based trading. 
Machine learning has been employed because there is a belief algorithm can be at least as good as human in entrying and
exiting positions. Such systems take different inputs but most of them are market-related.

 The majority of systems described
in the literature aim to maximize trading profits or some risk-adjusted measure such as the
Sharpe ratio.
Many attempts have been made to come up with a consistently profitable system and
inspiration has come from different fields ranging from fundamental analysis, econometric
modelling of financial markets, to machine-learning [5, 8]. Few attempts were successful
and those that seemed most promising often could not be used to trade actual markets
because of associated practical disadvantages. Among others these included large drawdowns
in profits and excessive switching behaviour resulting in very high transaction costs.
Professional traders have generally regarded those automated systems as far too risky in
comparison to the returns they were themselves able to deliver. Even if a trading model
was shown to produce an acceptable risk-return profile on historical data there was no
guarantee that the system would keep on working in the future. It would cease working
precisely at the moment it became unable to adapt to the changing market conditions.
This paper aims to deal with the above problems to obtain a usable, fully automated
and intelligent trading system. To accomplish this a risk management layer and a dynamic
optimization layer are added to a known machine-learning algorithm. The middle layer
manages risk in an intelligent manner so that it protects past gains and avoids losses by restraining
or even shutting down trading activity in times of high uncertainty. The top layer
dynamically optimizes the global trading performance in terms of a trader’s risk preferences
by automatically tuning the system’s hyper-parameters. While the machine-learning
system is designed to learn from its past trading experiences, the optimization overlay is
an attempt to adapt the evolutionary behaviour of the system and its perception of risk
to the evolution of the market itself. In the past an automated trading system based on 2
superimposed artificial intelligence algorithms was proposed [5]. This research departs from
a similar principle by developing a fully layered system where risk management, automatic
parameter tuning and dynamic utility optimization are combined. (For an earlier attempt
in this direction see [13].) The machine learning algorithm combined with the dynamic
optimization is termed adaptive reinforcement learning.
Section 2 of this paper briefly discusses the RRL machine-learning algorithm underlying
the trading system. Section 3 looks at the different layers of the trading system individually.
In 3.1 the modifications to the standard algorithm are set out and in 3.2 and 3.3 the risk
management and optimization layers are explained. The impressive performance of the
trading system is demonstrated in Section 4 and the final section concludes.
2
2 THE MACHINE-L


\section{Data}
Datasets used for the purpose of this workpaper are from the following databases:
\begin{itemize}
\item Thomson Reuters Tick Database
\item An aggregator tickdatabase
\end{itemize}

\section{Structure}

\section{First chapter}

The first part consists of the introduction to the problem. It outlines
the whole concept of the AI-related 
fields in finance.
It brought up historical background of finance and computer sciences, and its interdependency. Concretely, it 
includes the history of implementing first methods in early 80?s, the flash crash in October 1987, first 
recruitments of 'quants' on the Wall Street in the early 90?s.


\section{Second chapter}
This chapter starts with the critical discussion of models from finance. It includes both classic models, such as CAPM, a gold standard in equity research, and modern ones. The part is descriptive as it regards implicit pros and cons of financial models.

The latter part of the literature review is specifically about algorithmic trading and the methodology of other similar researches, e.g. Sakowski et al. (2013).
The last subchapter is about machine learning algorithms that are used in trading.
(8-10 pages)

\section{Third chapter}
The third chapter will start with goals of the research. I want to make it clear why this work is important. It was partially discussed in Problem part of this text. This master?s thesis is to find an application of the Reinforcement Learning for financial data. This part will contain hypotheses which are as follows:
  Algorithms based on artificial intelligence can be fruitful for investors by outperforming benchmarks in both risk and return;
Better performance turns out to be true in high-frequency trading and on longer period intervals;
Algorithms can learn how to spot overreacting on markets and choose the most under/overpriced security by exploiting time series analysis tools. 
(2 pages)

Methodology
This subchapter contains the description of methodology. It includes all formulas and steps that directed to final results. 
The algorithm itself will incorporate two environments:
  
\begin{itemize}
\item R - to incorporate libraries for machine learning
\item C++ - for code efficiency, it will help in improving performance in bottlenecks
\end{itemize}

The used algorithm is based on dynamic optimization approach. Besides a value function, there will be several indicators, e.g. RSI, which serve as a base for decision taking of the algorithm. The methodology will include transactional costs, so that the optimization is going to be implemented in a real-like environment

(10 pages)

The value function will be based by several statistics, such as the Sharpe and the Differental Sharpe Ratio to capture both risk and return.
As of now, I cannot enclose the exact form of formulas used in the research but I will provide them as soon as I write the proper code.
The output of my algorithm in R will be probably a set of positions ${-1,0,1}$, 
cumulated returns, and risk measures (not only the Sharpe ratio but also MD, MDD, the Sortino ratio, and 
                                      others).

How am I going to measure the efficiency of my code? I will implement several benchmarks ? the most logical 
choice is a buy-and-hold strategy on underlying asset (equities, equity-like securities). The second obvious 
choice is sort of random walk process. By this, I mean that a part of the algorithm will generate random 
values for a domain of ${-1,0,1}$ and these values will serve as a position. Of course, this benchmark will 
not include any transactional costs as this obvious that this extreme case would have an enormous cumulated 
transactional cost (position would change in $frac{2}{3}$ of states).
When I have the data I am going to discuss my results with other works. Outline the possible directions of 
future research papers on the issue: What can be implemented? What additionally can be done and measured?
Fourth chapter
This part consists of conclusions. Once again, I will write what have been done in this master?s thesis, and 
everything that conclusions should contain.


\addcontentsline{toc}{chapter}{Introduction}
% 


\chapter{FX Market}

\begin{knitrout}
\definecolor{shadecolor}{rgb}{0.969, 0.969, 0.969}\color{fgcolor}\begin{kframe}


{\ttfamily\noindent\itshape\color{messagecolor}{\#\# \\\#\# \\\#\# processing file: chapters/subchapters/1\_2\_FX\_Market\_Organization.Rnw}}\begin{verbatim}
## 
  |                                                                       
  |                                                                 |   0%
  |                                                                       
  |.............                                                    |  20%
##    inline R code fragments
\end{verbatim}


{\ttfamily\noindent\itshape\color{messagecolor}{\#\# \\\#\# \\\#\# processing file: chapters/subchapters/../../masters\_thesis.Rnw}}

{\ttfamily\noindent\itshape\color{messagecolor}{\#\# Quitting from lines 2-24 (chapters/subchapters/../../masters\_thesis.Rnw)}}

{\ttfamily\noindent\bfseries\color{errorcolor}{\#\# Error in parse\_block(g[-1], g[1], params.src): duplicate label 'r'}}\end{kframe}

\end{knitrout}



\chapter{Financial models}


% !Rnw root = ../../masters_thesis.Rnw



The following chapter introduces articles that correspond with the subject of the current thesis and are considered as fundamentals of modern finance.
Specifically, the beginning contains financial market models. The next subchapter includes basic investment effectiveness indicators that implicitly or explicitly result from the fundamental formulas from the first subchapter.

\section{Selected financial market models and theory}


Works considered as a fundament of quantitative finance and investments are Sharpe \parencite{Sharpe1964}, Lintner \parencite{Lintner1965}, and Mossin \parencite{Mossin1966}. All these authors, almost simultaneously, formulated Capital Asset Pricing Model (CAPM) that describes dependability between rate of return and its risk, risk of the market portfolio, and risk premium.
Assumptions in the model are as follows:
\begin{itemize}
\item Decisions in the model regard only one period,
\item Market participants has risk aversion, i.e. their utility function is related with plus sign to rate of return, and negatively to variance of portfolio rate of return,
\item Risk-free rate exists,
\item Asymmetry of information non-existent,
\item Lack of speculative transactions,
\item Lack of transactional costs, taxes included,
\item Market participants can buy a fraction of the asset,
\item Both sides are price takers,
\item Short selling exists,
\end{itemize}
Described by the following model formula is as follows:
\begin{equation}
E(R_P)=R_F+\frac{\sigma_P}{\sigma_M}\times[E(R_M)-R_F]
\end{equation}
where:
\begin{itemize}
\item $E(R_P )$ -- the expected portfolio rate of return,
\item $E(R_M)$ -- the expected market rate of return,
\item $R_F$ -- risk-free rate,
\item $\sigma_P$ -- the standard deviation of the rate of return on the portfolio,
\item $\sigma_M$ -- the standard deviation of the rate of return on the market portfolio.
\end{itemize}
$E(R_P)$ function is also known as Capital Market Line (CML). Any portfolio lies on that line is effective, i.e. its rate of return corresponds to embedded risk.
The next formula includes all portfolios, single assets included. It is also known as Security Market Line (SML) and is given by the following equation:
\begin{equation} \label{eq:erl}
E(R_i)=R_F+\beta_i\times[E(R_M)-R_F]
\end{equation}
where:
\begin{itemize}
\item $E(R_i)$ -- the expected $i$-th portfolio rate of return,
\item $E(R_M)$ -- the expected market rate of return,
\item $R_F$ -- risk-free rate,
\item $\beta_i$ -- Beta factor of the $i$-th portfolio.
\end{itemize}

\section{The Modern Portfolio Theory}
The following section discuss the Modern Portfolio Theory developed by Henry Markowitz \parencite{Markowitz1952}. The author introduced the model in which the goal
(investment criteria) is 
not only to maximize the return but also to minimize the variance. He claimed that by combining assets in different composition it is possible to obtain the 
portfolios with the same return but different levels of risk. The risk reduction is possible by diversification, i.e. giving proper weights for each asset 
in the portfolio. Variance of portfolio value can be effectively reduced by analyzing mutual relations between returns on assets with use of methods in statistics
(correlation and covariance matrices). It is important to say that any additional asset in portfolio reduces minimal variance for a given portfolio 
but it is the correlation what really impacts the magnitude.
The Markowitz theory implies that for any assumed expected return there is the only one portfolio that minimizes risk. Alternatively, there is only one portfolio 
that maximizes return for the assumed risk level. The important term, which is brought in literature, is the effective portfolio, i.e. the one that meets conditions
above.
The combination of optimal portfolios on the bullet.

Bullet figure

The Markowitz concept is determined by the assumption that investors are risk-averse. This observation is described by the following formula:

\begin{equation}
E(U)<U(E(X))
\end{equation}
where:
\begin{itemize}
\item $E(U)$ -- the expected value of utility from payoff;
\item $U(E(X))$ -- utility of the expected value of payoff.
\end{itemize}
The expected value of payoff is given by the following formula:
\begin{equation}
E(U)=\sum_{i=1}^{n}\pi_iU(c_i)
\end{equation}
where:
\begin{itemize}
\item $\pi_i$ -- probability of the $c_i$ payoff,
\item $U(c_i)$ -- utility from the $c_i$ payoff.
\end{itemize}
One of the MPT biggest flaws is the fact that it is used for ex post analysis. Correlation between assets changes overtime so results must be recalculated. Real portfolio risk may be underestimated. Also, time window can influence the results.

\section{The efficient market hypothesis}

In 1965, Eugene Fama introduced the efficient market term \parencite{Fama1965}. Fama claimed that an efficient market is the one that instanteneously discounts the new information arrival in market price of a given asset. Because this definition applies to financial markets, it had determined the further belief that it is not possible to beat the market because assets are perfectly priced. Also, if this hypothesis would be true, market participants cannot be better or worse. Their portfolio return would be a function of new, unpredictable information. In that respect, the only role of an investor is to manage his assets so that the risk is acceptable. 


% !Rnw root = ../../masters_thesis.Rnw

%\SweaveOpts{concordance=TRUE}


\section{Selected investment performance measures}

Introduced articles does not include any indicator that would explicitly measure portfolio management effectiveness. 
Equations that result from the authors' work are important because some of further developed measures are CAPM-based. 
The most known are the Sharpe ratio, the Treynor ratio, and the Jensen's alpha. Popularity of these indicator comes from the fact that 
they are easy to understand for the average investor. \parencite{Marte2012}
In \parencite{Sharpe1966}, the author introduced the $\frac{R}{V}$ indicator, also known as the Sharpe Ratio ($S$), which is given by the following formula:
\begin{equation}
S_i=\frac{E(R_i-R_F)}{\sigma_i}
\end{equation}
where: 
\begin{itemize}
\item $R_i$ -- the $i$-th portfolio rate of return,
\item $R_F$ -- risk-free rate
$\sigma_i$ -- the standard deviation of the rate of return on the $i$-th portfolio.
\end{itemize}
Treynor (Treynor1965) proposed other approach in which denominator includes $\beta_i$ instead of $\sigma_i$. The discussed formula is given by:
\begin{equation}
T_i=\frac{R_i-R_F}{\beta_i}
\end{equation}
where:
\begin{itemize}
\item $R_i$ -- the $i$-th portfolio rate of return,
\item $R_F$ -- Risk-free rate
\item $\beta_i$ -- Beta factor of the $i$-th portfolio.
\end{itemize}
Both indicators, i.e. $S$ and $T$ are relative measures. Their value should be compared with a benchmark to determine if a given portfolio is well-managed. If they are
higher (lower), it means that analyzed portfolios were better (worse) than a benchmark.
The last measure, very popular among market participants, is the Jensen's alpha. It is given as follows:
\begin{equation}
\alpha_i=R_i-[R_F+\beta_i\times(R_M-R_F )]
\end{equation}
where:
\begin{itemize}
\item $R_i$ -- the $i$-th portfolio rate of return,
\item $R_F$ -- Risk-free rate
\item $\beta_i$ -- Beta factor of the $i$-th portfolio.
\end{itemize}
The Jensen's alpha is an absolute measure and is calculated as the difference between actual and CAPM model-implied rate of return. The greater the value is,
the better for the $i$-th observation.

The differential Sharpe ratio - this measure is a dynamic extension of Sharpe ratio. By using the indicator, it can be possible to capture a marginal impact of return at time t on the Sharpe Ratio. The procedure of computing it starts with the following two formulas:
\begin{equation}
A_n=\frac{1}{n}R_n+\frac{n-1}{n}A_{n-1}
\end{equation}
\begin{equation}
B_n=\frac{1}{n}R_n^2+\frac{n-1}{n}B_{n-1}
\end{equation}
At $t=0$ both values equal to 0. They serve as the base for calculating the actual measure - an exponentially moving Sharpe ratio on $\eta$ time scale.
\begin{equation}
S_t=\frac{A_t}{K_\eta\sqrt{B_t-A_t^2}}
\end{equation}
where:
\begin{itemize}
\item $A_t=\eta R_t+(1-\eta)A_{t_1} $ 
\item $B_t=\eta R_t^2+(1-\eta)B_{t_1} $ 
\item $K_\eta=(\frac{1-\frac{\eta}{2}}{1-\eta})$
\end{itemize}

Using of the differential Sharpe ratio in algorithmic systems is highly desirable due to the following facts \parencite{Moody1997}:
\begin{itemize}
\item Recursive updating - it is not needed to recompute the mean and standard deviation of returns every time the measure value is evaluated. 
Formula for $A_t$ ($B_t$) enables to very straightforward calculation of the exponential moving Sharpe ratio, just by updating for  $R_t$ ($R_t^2$)
\item Efficient on-line optimization - the way the formula is provided directs to very fast computation of the whole statistic with just updating the most recent values
\item Interpretability - the differential Sharpe ratio can be easily explained, i.e. it measures how the most recent return affect the Sharpe ratio (risk and reward).
\end{itemize}


The drawdown is the measure of the decline from a historical peak in an asset.
The formula is given as follows:

\begin{equation}
D(T)=\max\{max_{0, t\in (0,T)} X(t)-X(\tau)\}
\end{equation}


The Sterling ratio (SR)


The maximum drawdown (MDD) at time $T$ is the maximum of the Drawdown over the asset history. The formula is given as follows:

\begin{equation}
MDD(T)=\max_{\tau\in (0,T)}[\max_{t\in (0,\tau)} X(t)-X(\tau)]
\end{equation}

\chapter{Research}




\section{Research Objective}

The primary research goal is to evaluate the Reinforcement Learning-based algorithm for multiasset trading.
The main idea behind the algorithm deployment is that it can systematically outperform benchmarks 
in terms of both risk and return. 
The trading system will be able to spot non-trivial patterns, faster than human, and exploit them.
Research will
The goal of this project is to assess the possibility of using Reinforcement Learn-
ing to create a trading agent which is capable of nding persistent similarities in 
nancial time series and which learns how to dene and exploit deviations from
the expected, prevalent behaviour. We design and compare two approaches, a basic
approach based on Monte Carlo Control and an extended approach based on Qlearn-
ing and value function approximation. The rst approach is aimed to provide more
interpretability, the second approach is to provide better performance. We assess
the outcome by the trading performance of the two agents on two weeks of out-of-
sample fx market test data with one minute granulation. We set two benchmarks by
which we measure trading performance. The percentage return and the Sharpe ratio
of the trades the agent engages in should be higher than the return of a buy-and-
hold strategy of either of the underlying cointegrated assets. A secondary goal is
to draw conclusions about the interactions in parameter settings of the pair trading
frame work and how they in
uence protability of the pair trade. The objectives
are summarized in the following list:
The objectives are as follows:

Design and deployment - this part 


\addcontentsline{toc}{chapter}{Bibliography}
\printbibliography


\listoffigures
\listoftables

\end{document}

\begin{knitrout}
\definecolor{shadecolor}{rgb}{0.969, 0.969, 0.969}\color{fgcolor}\begin{kframe}
\begin{alltt}
\hlkwd{system}\hlstd{(}\hlkwd{paste}\hlstd{(}\hlstr{"biber"}\hlstd{,} \hlkwd{sub}\hlstd{(}\hlstr{"\textbackslash{}\textbackslash{}.Rnw$"}\hlstd{,} \hlstr{""}\hlstd{,} \hlkwd{current_input}\hlstd{())))}
\end{alltt}


{\ttfamily\noindent\color{warningcolor}{\#\# Warning: running command 'biber masters\_thesis-knitr' had status 127}}\end{kframe}
\end{knitrout}
